\documentclass[12pt, a4paper]{article}
\usepackage[utf8]{inputenc}
\usepackage{soul}
\usepackage{multirow}
\usepackage{multicol}
%% \usepackage[margin=1in]{geometry}
\usepackage[american]{babel}


\usepackage{array}
\newcolumntype{L}[1]{>{\raggedright\let\newline\\\arraybackslash\hspace{0pt}}m{#1}}

\title{Project Proposal}
\author{Dexter Gerig, Dennis Pham, Amey Shukla}
\date{\today}

\begin{document}

\begin{titlepage}
  \maketitle
\end{titlepage}

\section{Summary}
The fuzzer will be used to attempt to find bugs in existing software and help provide testcases for fixing issues faster for developers. It works by mutating known working samples and sending them to the application/library to be fuzzed. It then detects any crashes/hangs and reports the result. The fuzzer attempts to uniquly identify each problem found with a trace of the execution path that resulted in it to attempt to prevent reporting the same bug multiple times.

\section{Scope}
The fuzzer will need to have mutators that take input and preform actions on it. These actions are designed to bring about undefined behavior in the application/library being fuzzed. There will be several mutators that each preform a specific type of action like random bit manipulation, string extension, header corruption, etc... After this the program/library will be fed the input and traced to see if it fails. If a failure occurs a debugger will be attached and a stack trace retrieved. The stack trace can be used to determine if this is a unique crash/hang or not and if the issue appears to have been already found, if the issue has been encountered before then it will discard the result. If the issue is new the debugged application will be analyized to detect the type of crash/hang and attempt to assign a score to the problem to indicate severity. The results along with a copy of the input fed to the application will be written out and labeled so that they can be examined later by the developer to determine the cause of the problem.

\begin{itemize}
  \item Fulfiling Jira Style Tickets on Gitlab.
  \item Team will research known bugs and exploits caught by existing production Fuzzers to compare against.
  \item a prototype will be made to find exploits in open source production applications such as vlc and libjpeg.
  \item results of each run and build will be collected and graphed using Matplotlib.
\end{itemize}

\pagebreak
\section{Legal and Ethics}
This tool will not be used against any library or application that prevents the disassembly or reverse engineering of itself. There are no other predicted legal problems for using a fuzzer however, there are slight ethical issues that could crop up. Mainly that of what is done with the bugs that are found. Ethically the bugs should be reported the company/developer of the product so that the bug can be fixed. Failing to disclose the bugs to the developer could negativly impact anybody using the product if the bugs are found by malicious users or accidently disclosed publicly before fixed.

\section{Team Analysis}
\subsection{Software}
All members are currently employed by a company that requires an intermediate level of coding skill. Development
environemnts are handled since everyone on the team uses Linux and is comfortable with spinning up Virtual machines to
do their daily work. Debug tools, configuration management, and protocols knowledge are adequate but are also not
critical for the development of the tool.

\subsection{Electrical}
This is not needed for the project.

\subsection{Computing}
Each member is able to jump between operating systems to support some level of cross compatibility and is able to set up
environments on small board computers such as the Raspberry Pi and Arduino.

\subsection{Project Management}
One student is able to lead as a project manager at a basic level, due to the small size of the team. Dennis Pham is the project manager and is responsible for ensuring tasks are divided to all members in the teams and making sure the project is in progress to completion. Dexter Gerig is the technical lead and is responsible for specifying technical tasks that need to be done by the team. Amey Shukla is responsible for taking minutes during the meetings and making tickets.

\subsection{Potential Risks and anticipated roadblocks}
With only one person able to do project management the group might fall behind in milestones, and the lack of hardware
experience might interfere with the quality of the tool if it's required to work on embedded system with strict
constraints on how the sensors work.

\section{Tasks and Deliverables}
The target platform for the tool will be ARM Libraries and Binaries but the application itself will run on x86.
Its features include but are not limited to.

\begin{multicols}{2}
  \begin{itemize}
    \item Bit Flipper
    \item String Extender
    \item Number Mutator
    \item Truncator
    \item Library Input
    \item Application Input
  \end{itemize}
\end{multicols}

\subsection{Project Management}
\begin{table}[h]
  \begin{tabular}{|L{5cm}|L{10cm}|}\hline
    \textbf{Task/Responsibility} & \textbf{Notes} \\ \hline
    \multirow{3}{*}{\textbf{Jira Style Tickets}} & Keeps track of features.\\
                                                 & Keeps track of time spent on features.\\
                                                 & Keeps track of known issues and bugs.\\ \hline
    \textbf{Research} & Google's AFL will be used as a baseline to find known issues within open source projects.\\\hline
    \textbf{Logging caught vulnerabilities} & Bugs will be logged and compared to fuzzers run on the same target.\\\hline
    \textbf{Benchmarks} & Applicants with known issues will be fuzzed to test performance and profile for optimizations.\\\hline
    \textbf{Logging Bugs} & Bugs will be logged in the gitlab repository to be fixed before final release.\\\hline
  \end{tabular}
\end{table}

\pagebreak
\subsection{Deliverables + Team Schedule}
\begin{table}[h]
  \begin{tabular}{|L{5cm}|L{10cm}|}
    \hline
    \textbf{Week} & \textbf{Notes} \\ \hline
    \textbf{Week 1} & Investigate other open source software to find a project. \\ \hline
    \textbf{Week 2} & Find a name for our project and assign roles to each team member. \\ \hline
    \textbf{Week 3} & Setup boilerplate repository and investigate other fuzzers. \\ \hline
    \textbf{Week 4} & Work on fuzzer input mutation rig and start work on binary analysis. \\ \hline
    \multirow{2}{*}{\textbf{Week 5}} & * Compile list of mutators to make and divide up work. \\
                                     & * Work on binary analysis and determine if ARM emulator is viable. \\
                                     & * Work on mutators and start integrating them together. \\ \hline
    \multirow{2}{*}{\textbf{Week 6}} & * Begin running test cases and attempt to trigger known bugs. \\
                                     & * Work on creating rig for being able to fuzz ARM binaries with better code analysis (x86 is too difficult). \\
                                     & * Decide on libraries/applications worth fuzzing. \\ \hline
    \textbf{Week 7} & Begin creating test cases for targets. \\ \hline
    \multirow{2}{*}{\textbf{Week 8-16}} & * Setup and run fuzzer on targets (takes a long time). \\
                                        & * Look for opportunities to create more mutators to improve fuzzer. \\ \hline
  \end{tabular}
\end{table}

\section{Acceptance Crietia}
\textbf{Given} a program is set as a target to the fuzzer. \\
\textbf{When} fuzzer mutates working samples and send them to the target.\\
\textbf{Then} existing bugs and crashes are reported with the trace of execution path.\\


\section{Required Skills}
The team should have heavy knowledge of C programming and minor x86+ARM assembly. Extra skills that will help are experience in writing emulators and heavy knowledge of ARM assembly for designing the binary analysis. All members of the team have the required knowledge and skills needed to complete the project.

\end{document}
